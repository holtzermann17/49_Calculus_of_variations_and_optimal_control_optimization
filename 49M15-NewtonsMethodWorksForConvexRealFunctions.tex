\documentclass[12pt]{article}
\usepackage{pmmeta}
\pmcanonicalname{NewtonsMethodWorksForConvexRealFunctions}
\pmcreated{2013-03-22 15:41:28}
\pmmodified{2013-03-22 15:41:28}
\pmowner{stevecheng}{10074}
\pmmodifier{stevecheng}{10074}
\pmtitle{Newton's method works for convex real functions}
\pmrecord{7}{37634}
\pmprivacy{1}
\pmauthor{stevecheng}{10074}
\pmtype{Theorem}
\pmcomment{trigger rebuild}
\pmclassification{msc}{49M15}
\pmclassification{msc}{65H05}
\pmclassification{msc}{26A06}
\pmsynonym{Newton's method works for concave real functions}{NewtonsMethodWorksForConvexRealFunctions}
%\pmkeywords{Newton's method}
%\pmkeywords{Newton-Raphson method}

\usepackage{amssymb}
\usepackage{amsmath}
\usepackage{amsfonts}
\usepackage{amsthm}

% used for TeXing text within eps files
%\usepackage{psfrag}
% need this for including graphics (\includegraphics)
\usepackage{graphicx}
% making logically defined graphics
%%%\usepackage{xypic}

% define commands here
\newcommand{\real}{\mathbb{R}}

\providecommand{\abs}[1]{\lvert#1\rvert}
\providecommand{\absW}[1]{\left\lvert#1\right\rvert}
\providecommand{\absB}[1]{\Bigl\lvert#1\Bigr\rvert}
\providecommand{\defnterm}[1]{\emph{#1}}

\newtheorem{thm}{Theorem}
\begin{document}
\PMlinkescapeword{argument}
\PMlinkescapeword{simple}

\begin{thm}
\label{thm:main}
Let $f\colon I \to \real$ be a convex differentiable function
on an interval $I \subseteq \real$, with at least one root.
Then the following sequence $\{ x_n \}$
obtained from Newton's method,
\[
x_{n+1} = x_n - \frac{f(x_n)}{f'(x_n)}\,,
\]
will converge to a root of $f$, provided that
$f'(x_0) \neq 0$ and $x_1 \in I$ for the given starting point $x_0 \in I$.
\end{thm}

Obviously, ``convex'' can be replaced by ``concave'' in Theorem \ref{thm:main}.

The proof will proceed in several steps.  First, a simple converse result:

\begin{thm}
\label{thm:converse}
Let $f\colon I \to \real$ be a differentiable convex function,
and $\{x_n\}$ be the sequence in Theorem \ref{thm:main}.
If it is convergent to a number $a \in I$,
then $a$ is necessarily a root of $f$.

\begin{proof}
We have
\[
0 = \lim_{n \to \infty} f'(x_n) \cdot (x_n - x_{n+1}) = 
\lim_{n \to \infty} f(x_n) = f(a)\,.
\]
(The first limit is zero by local boundedness of $f'$ at $a$,
which follows from $f'$ being finite and monotone\footnote{
Actually, a differentiable convex function must necessarily have a continuous derivative,
for the derivative is increasing, and it cannot have any jump discontinuities by 
\PMlinkname{Darboux's Theorem}{DarbouxsTheorem}.}.)
\end{proof}
\end{thm}

\section*{Proof of Theorem \ref{thm:main}}

\subsection*{Case A: when $I = \real$, $f' > 0$, and $f(x_0) \geq 0$}
We claim that whenever $f(x_n) \geq 0$, then $f(x_{n+1}) \geq 0$ too.
Recall that the graph of a convex function $f$ always lies 
above any of its tangent lines.
So in particular, the point $(x_{n+1}, f(x_{n+1}))$ is higher than the point
$(x_{n+1}, 0)$, which is on the tangent line by definition.
By induction, we have $f(x_n) \geq 0$ for all $n$.

We also have $x_{n+1} \leq x_n$ for all $n$.
By hypothesis the slope $f'(x_n)$ of the tangent line is positive, 
and $(x_n, f(x_n))$ lies above the horizontal axis.
Then the intersection point of the tangent line and the horizontal axis,
which is $(x_{n+1}, 0)$, must be to the left of $(x_n, 0)$.

So the sequence $\{ x_n \}$ is decreasing.  It converges to a real number or
diverges to $-\infty$.  
If the limit is a real number, by Theorem \ref{thm:converse} that number
is a root of $f$, and we are done.

If the limit is $-\infty$, then we must have $f(x_n) > 0$
for all $x_n$. Hence $f(x) > 0$ for all $x$, as $f$ is monotone
and $x_n$ can be arbitrarily large negative.
In other words, $f$ has no root to begin with.

\begin{figure}[!htb]
\begin{center}
\includegraphics{convex-newton.1.eps}
\end{center}
\caption{Case A: $\{ x_n \}$ is monotonically decreasing}
\end{figure}

\subsection*{Case B: when $I = \real$, $f' < 0$, and $f(x_0) \geq 0$}
This situation is the (left-right) mirror of Case A, 
except that because the slope of the curve is negative,
the sequence $\{ x_n \}$ is increasing this time.  We still have $x_n \geq 0$,
so the sequence either converges to a root of $f$, or diverges to $+\infty$ (in which case $f$ has no root). 

\begin{figure}[!htb]
\begin{center}
\includegraphics{convex-newton.2.eps}
\end{center}
\caption{Case B: $\{ x_n \}$ is monotonically increasing}
\end{figure}

\subsection*{Case C: when $I = \real$, $f'(x_0) > 0$, and $f(x_0) \geq 0$}
To begin, note that the first part of Case A, asserting that $f(x_n) \geq 0$
for all $n$, still goes through only assuming $f'(x_n) \neq 0$
--- in which case the sequence would not be defined after the point $x_n$.
We show that, in fact, $f'(x_n) > 0$ for all $n$,
so the rest of Case A goes through. 

Suppose $n$ is the smallest integer
for which $f'(x_{n+1}) \leq 0$.  
Also assume $f(x_n), f(x_{n+1}) \neq 0$, otherwise we would be done anyway.
The reasoning in Case A shows $x_{n+1} < x_n$.

\begin{figure}[!htb]
\begin{center}
\includegraphics{convex-newton.3.eps}
\end{center}
\caption{Case C: a tangent line separates $f$ from the horizontal axis}
\end{figure}

The function $f$ is strictly positive on $[x_{n+1}, x_n]$,
for the graph of $f$ lies above the tangent line segment extending
from $(x_{n+1}, 0)$ to $(x_n, f(x_n))$, which in turn lies 
strictly above the horizontal axis.
Since $f$ is decreasing to the left of $[x_{n+1}, x_n]$
and increasing to the right, 
$f$ never touches the horizontal axis anywhere.
This contradicts our hypothesis that $f$ has roots.

\subsection*{Case D: when $I = \real$, $f'(x_0) < 0$, and $f(x_0) \geq 0$}
This situation is the (left-right) mirror 
of Case C.  The same argument shows that $f'(x_n) < 0$
for all $n$ (or $f(x_n) = 0$ for some $n$), and the
argument of Case B applies.

\subsection*{Case E: for general intervals $I$}
The only concern is that the iteration $\{ x_n \}$
might go outside the interval $I$.  We show that it does not.

Suppose $f(x_0) \geq 0$.
Then as we have proved, $f(x_n) \geq 0$ for all $n$.
Suppose $x_n \in I$ but $x_{n+1} \notin I$.
Then $f$ has no root, because the graph of $f$ lies above
the tangent line from $(x_n, f(x_n))$ to $(x_{n+1}, 0)$
which lies above the horizontal axis.

The \emph{limit} of $x_n$ could conceivably go just outside
the interval $I$.
But this case is similar to the case $x_n \to \pm \infty$
already considered in Case A:
$f$ possesses no root then.


Finally, suppose $f(x_0) < 0$, the case we have ignored all along.
Our hypotheses assume that $f(x_1)$ is defined.
We immediately have $f(x_1) \geq 0$, 
since the graph of $f$ lies above the tangent line
from $(x_0, f(x_0))$ to $(x_1, 0)$.
But this just brings us back to the other cases.

\begin{figure}[!htb]
\begin{center}
\includegraphics{convex-newton.4.eps}
\includegraphics{convex-newton.5.eps}
\end{center}
\caption{Case E: $f(x_1)$ is always $\geq 0$}
\end{figure}

\section*{Examples}

Newton's method for finding square roots of positive numbers $\alpha$
(which reduces to the ancient divide-and-average method) always
converges, for the target function $f(x) = x^2 - \alpha$ is convex.

\begin{thebibliography}{3}
\bibitem{Spivak}
Michael Spivak. {\it Calculus}, third edition. Publish or Perish, 1994.
\end{thebibliography}
%%%%%
%%%%%
\end{document}
