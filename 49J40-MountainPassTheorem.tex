\documentclass[12pt]{article}
\usepackage{pmmeta}
\pmcanonicalname{MountainPassTheorem}
\pmcreated{2013-03-22 15:19:19}
\pmmodified{2013-03-22 15:19:19}
\pmowner{ncrom}{8997}
\pmmodifier{ncrom}{8997}
\pmtitle{mountain pass theorem}
\pmrecord{8}{37128}
\pmprivacy{1}
\pmauthor{ncrom}{8997}
\pmtype{Theorem}
\pmcomment{trigger rebuild}
\pmclassification{msc}{49J40}
%\pmkeywords{Mountain Pass}

\endmetadata

% this is the default PlanetMath preamble.  as your knowledge
% of TeX increases, you will probably want to edit this, but
% it should be fine as is for beginners.

% almost certainly you want these
\usepackage{amssymb}
\usepackage{amsmath}
\usepackage{amsfonts}

% used for TeXing text within eps files
%\usepackage{psfrag}
% need this for including graphics (\includegraphics)
%\usepackage{graphicx}
% for neatly defining theorems and propositions
%\usepackage{amsthm}
% making logically defined graphics
%%%\usepackage{xypic}

% there are many more packages, add them here as you need them

% define commands here
\begin{document}
Let $X$ a real Banach space and $F \in C^{1}(X,\mathbb{R})$. Consider $K$ a compact metric space, and $K^{*} \subset K$ a closed nonempty subset of $K$. If $p^{*} : K^{*} \rightarrow X$ is a continuous mapping, set 
\[
\mathcal{P} = \{ p \in C(K, \, X); \; p = p^{*} \; \textrm{on } K^{*} \}.
\]

Define 
\[
c = \inf_{p \in \mathcal{P}} \max_{t \in K} F(p(t)).
\]

Assume that
\begin{equation}\label{eq1}
c > \max_{t \in K^{*}} F(p^{*}(t)).
\end{equation}
Then there exists a sequence $(x_{n})$ in $X$ such that
\begin{enumerate}
\item[(i)] $\lim\limits_{n \rightarrow \infty} F(x_{n}) = c$;

\item[(ii)] $\lim\limits_{n \rightarrow \infty} \| F'(x_{n})\|  = 0$.
\end{enumerate}

The name of this theorem is a consequence of a simplified visualization for the objects from theorem. If we consider the set $K^{*} = \{ A, \, B\}$, where $A$ and $B$ are two villages, $\mathcal{P}$ is the set of all the routes from $A$ to $B$, and $F(x)$ represents the altitude of point $x$; then the assumption (\ref{eq1}) is equivalent to say that the villages $A$ and $B$ are separated with a mountains chain. So, the conclusion of the theorem tell us that exists a route between the villages with a minimal altitude. With other words exists a  ``mountain pass'' .
%%%%%
%%%%%
\end{document}
