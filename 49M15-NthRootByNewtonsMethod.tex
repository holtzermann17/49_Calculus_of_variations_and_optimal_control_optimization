\documentclass[12pt]{article}
\usepackage{pmmeta}
\pmcanonicalname{NthRootByNewtonsMethod}
\pmcreated{2013-03-22 19:09:38}
\pmmodified{2013-03-22 19:09:38}
\pmowner{pahio}{2872}
\pmmodifier{pahio}{2872}
\pmtitle{nth root by Newton's method}
\pmrecord{10}{42066}
\pmprivacy{1}
\pmauthor{pahio}{2872}
\pmtype{Example}
\pmcomment{trigger rebuild}
\pmclassification{msc}{49M15}
\pmclassification{msc}{65H05}
\pmclassification{msc}{26A06}
\pmsynonym{cube root of 2}{NthRootByNewtonsMethod}
\pmrelated{NthRoot}

% this is the default PlanetMath preamble.  as your knowledge
% of TeX increases, you will probably want to edit this, but
% it should be fine as is for beginners.

% almost certainly you want these
\usepackage{amssymb}
\usepackage{amsmath}
\usepackage{amsfonts}

% used for TeXing text within eps files
%\usepackage{psfrag}
% need this for including graphics (\includegraphics)
%\usepackage{graphicx}
% for neatly defining theorems and propositions
 \usepackage{amsthm}
% making logically defined graphics
%%%\usepackage{xypic}

% there are many more packages, add them here as you need them

% define commands here

\theoremstyle{definition}
\newtheorem*{thmplain}{Theorem}

\begin{document}
\PMlinkescapeword{formula}
The Newton's method is very suitable for computing approximate values of higher $n^{\mathrm{th}}$ \PMlinkname{roots}{NthRoot} of positive numbers (and odd roots of negative numbers!).\\

The general recurrence formula 
$$x_{k+1} \;=\; x_k-\frac{f(x_k)}{f'(x_k)}$$
of the method for determining the zero of a function $f$, applied to 
$$f(x) \;:=\; x^n\!-\!\alpha$$
whose zero is $\sqrt[n]{\alpha}$, reads
\begin{align}
x_{k+1} \:=\; \frac{1}{n}\left[(n\!-\!1)x_k+\frac{\alpha}{x_k^{n-1}}\right].
\end{align}
For a radicand $\alpha$, beginning from some initial value $x_0$ and using (1) repeatedly with successive values of 
$k$, one obtains after a few steps a sufficiently accurate value of $\sqrt[n]{\alpha}$ if $x_0$ was not very far from the searched root.\\

Especially for cube root $\sqrt[3]{\alpha}$, the formula (1) is
\begin{align}
x_{k+1} \:=\; \frac{1}{3}\left[2x_k+\frac{\alpha}{x_k^2}\right].
\end{align}
For example, if one wants to compute $\sqrt[3]{2}$ and uses\, $x_0 = 1$, already the fifth step gives 
$$x_5 \;=\; 1.259921049894873$$
which \PMlinkescapetext{contains fifteen right} decimals.

%%%%%
%%%%%
\end{document}
